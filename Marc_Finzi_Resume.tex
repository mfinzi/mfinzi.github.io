%% start of file `template.tex'.
%% Copyright 2006-2010 Xavier Danaux (xdanaux@gmail.com).
%% Copyright 2010-2011 Mark Liu (markwayneliu@gmail.com).
%
% This work may be distributed and/or modified under the
% conditions of the LaTeX Project Public License version 1.3c,
% available at http://www.latex-project.org/lppl/.

\documentclass[10pt,a4paper,sans]{moderncv}

% moderncv themes
\moderncvstyle{classic}
\moderncvcolor{blue}

% character encoding
\usepackage[utf8]{inputenc}                   % replace by the encoding you are using
\usepackage{url}
%\usepackage{hyperref}
% adjust the page margins
\usepackage[scale=.9,top=1in]{geometry}
%=\setlength{\hintscolumnwidth}{3cm}                        % if you want to change the width of the column with the dates
%\AtBeginDocument{\setlength{\maketitlenamewidth}{6cm}}  % only for the classic theme, if you want to change the width of your name placeholder (to leave more space for your address details
%\AtBeginDocument{\recomputelengths}                     % required when changes are made to page layout lengths

% personal data
\firstname{Marc}
\familyname{Finzi}

\address{136E Veterans Place, Ithaca, NY} % optional, remove the line if not wanted
\mobile{(240) 461-2365} % optional, remove the line if not wanted
\email{maf388@cornell.edu} % optional, remove the line if not wanted
\homepage{mfinzi.github.io}
%\extrainfo{\url{http://markliu.me}} % optional, remove the line if not wanted
\renewcommand*{\namefont}{\fontsize{20}{29}\mdseries\upshape}
%\title{Computer Science}
% to show numerical labels in the bibliography; only useful if you make citations in your resume
%\makeatletter
%\renewcommand*{\bibliographyitemlabel}{\@biblabel{\arabic{enumiv}}}
%\makeatother


\usepackage{bibentry}
\makeatletter\let\saved@bibitem\@bibitem\makeatother
%\usepackage[colorlinks=true]{hyperref}
\makeatletter\let\@bibitem\saved@bibitem\makeatother


% hanging publications: 1st line starts at beginning of line, 
% further lines are placed a bit to the right
\usepackage{hanging}
\newcommand\publication[1]{%
    \smallskip\par\hangpara{1.5em}{1}\bibentry{#1}\smallskip
}

\nopagenumbers{} % uncomment to suppress automatic page numbering for CVs longer than one page
%----------------------------------------------------------------------------------
%            content
%----------------------------------------------------------------------------------
\begin{document}

%\maketitle
\maketitle
\vspace*{-10mm}
\section{Education}
\cventry{2017 - ????}{Ph.D. in Operations Research}{Cornell University}{Ithaca, NY}{}{}
\cventry{2013 - 2017}{B.S. Physics}{Harvey Mudd College}{Claremont, CA}{GPA: 3.7}{}
\cventry{2009 - 2013}{\textnormal{\emph{Georgetown Day School}}}{}{Washington, DC}{}{}

\vspace*{-1mm}

\section{Research Experience}

\cventry{2017 - Present}{Deep Learning Research}{Prof. Andrew Wilson's group}{Cornell}{PhD student}{
    \begin{itemize}
        \item Bested SoA performance on semi-supervised image classification benchmarks
    	\item Broadly interested in Deep Learning with smaller amounts of labeled data
    \end{itemize}
    }
\vspace*{3mm} 
\cventry{2015 - 2017}{Plasma Physics Research}{Prof. Tom Donnelly's lab}{Harvey Mudd College}{Undergraduate}{
    \begin{itemize}
        \item Led three-man HMC team to set up an experiment at UT Austin to test the theory of multipass Stochastic Heating using the high-power GHOST laser. This experiment required more than 20 optical elements inside the vacuum chamber, alignment to a $5\mu m$ focal spot size, and careful synchronization between two laser systems.
        \item Computer Vision based solution for identifying microspheres in SEM images, achieving 95\% accuracy.
        \item Experiment automation with LabView, NiDAQs, and ThorLab components.
    \end{itemize}
    }

\vspace*{3mm} 
\cventry{Summers 2014, 2015}{Electrical Engineering}{Alexander Kutyrev's lab}{NASA Goddard Space Flight Center}{Intern}{
	\begin{itemize}
		\item Brought concept of low cost, high precision (5mK), cryogenic temperature sensors based on the internal temperature response of commercially available transistors into practice.
		\item Designed and prototyped a PCB for sensor control and PID, interfacing with an external microcontroller over SPI.
	\end{itemize}
	}


\vspace*{3mm} 
\bibliographystyle{plain}
\bibliography{publications}
%\section*{Publications}
\renewcommand\refname{\begin{center}
                        \textbf{Reference}
                      \end{center}}
\publication{ben2018}
      \hspace*{30mm}  (\href{https://openreview.net/forum?id=rkgKBhA5Y7}{link}, \href{https://openreview.net/pdf?id=rkgKBhA5Y7}{pdf})
       % \bibentry{ben2018}
 \vspace*{-3mm}   
\section{Teaching Experience} 

\cventry{2017 - Present}{Teaching Assistant }{Cornell}{Ithaca, NY}{}{
    TA for Simulation Modeling and Analysis
    TA for Basic Engineering Probability and Statistics (ENGRD 2700)
}
\cventry{2015 - 2017}{Physics Academic Excellence Tutoring }{Harvey Mudd College}{Claremont, CA}{}{
    Tutoring for freshman and sophomore physics classes.
}


 \section{Technical Skills}
 \cvline{Relevant Coursework}{Advanced Machine Learning Systems, Computer Vision, Bayesian Machine Learning, Numerical Analysis for Data Science, Algorithms, Statistical Principles, Stochastic Processes, Differential Geometry, Algorithms, Programming Languages, Microprocessors.}
 \cvline{Programming Languages}{Python, C\texttt{++}, \LaTeX.}
 %\cvline{DL Frameworks}{Pytorch, Tensorflow}
% \cvline{Software}
% {GitHub, Eagle, Visual Studio.}
% \cvline{Hardware Experience}
% { \begin{itemize}
%         \item Analog PCB Design%
%         \item SPI, ADC's
 %       \end{itemize}
 %   }
    \vspace*{-5.5mm}




%\vspace*{-1mm}
%\section{Extracurricular}
%\cventry{2014 - Present}{Programming Side Projects}{Claremont Colleges}{Claremont, CA}{}{
%	\begin{itemize}
%    \item A primitive Chess Engine written in C\texttt{++}
%	\item An interactive 1D Schrodinger Equation simulator in python to be used as a teaching tool
 %   \item A symplectic integrator for $n$ body gravitational system complete with 3D animation
%	\end{itemize}
%}


\end{document}
