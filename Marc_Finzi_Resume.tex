%% start of file `template.tex'.
%% Copyright 2006-2010 Xavier Danaux (xdanaux@gmail.com).
%% Copyright 2010-2011 Mark Liu (markwayneliu@gmail.com).
%
% This work may be distributed and/or modified under the
% conditions of the LaTeX Project Public License version 1.3c,
% available at http://www.latex-project.org/lppl/.

\documentclass[10pt,a4paper,sans]{moderncv}

% moderncv themes
\moderncvstyle{classic}
\moderncvcolor{blue}

% character encoding
\usepackage[utf8]{inputenc}                   % replace by the encoding you are using
\usepackage{url}
%\usepackage{hyperref}
% adjust the page margins
\usepackage[scale=.9,top=1in]{geometry}
%=\setlength{\hintscolumnwidth}{3cm}                        % if you want to change the width of the column with the dates
%\AtBeginDocument{\setlength{\maketitlenamewidth}{6cm}}  % only for the classic theme, if you want to change the width of your name placeholder (to leave more space for your address details
%\AtBeginDocument{\recomputelengths}                     % required when changes are made to page layout lengths

% personal data
\firstname{Marc}
\familyname{Finzi}

\address{New York City} % optional, remove the line if not wanted
\mobile{(240) 461-2365} % optional, remove the line if not wanted
\email{maf820@nyu.edu} % optional, remove the line if not wanted
\homepage{mfinzi.github.io}
%\extrainfo{\url{http://markliu.me}} % optional, remove the line if not wanted
\renewcommand*{\namefont}{\fontsize{20}{29}\mdseries\upshape}
%\title{Computer Science}
% to show numerical labels in the bibliography; only useful if you make citations in your resume
%\makeatletter
%\renewcommand*{\bibliographyitemlabel}{\@biblabel{\arabic{enumiv}}}
%\makeatother


\usepackage{bibentry}
\makeatletter\let\saved@bibitem\@bibitem\makeatother
%\usepackage[colorlinks=true]{hyperref}
\makeatletter\let\@bibitem\saved@bibitem\makeatother


% hanging publications: 1st line starts at beginning of line, 
% further lines are placed a bit to the right
\usepackage{hanging}
\newcommand\publication[1]{%
    \smallskip\par\hangpara{1.5em}{1}\bibentry{#1}\smallskip
}

\nopagenumbers{} % uncomment to suppress automatic page numbering for CVs longer than one page
%----------------------------------------------------------------------------------
%            content
%----------------------------------------------------------------------------------
\begin{document}
\vspace*{-10mm}
%\maketitle
\maketitle
\vspace*{-10mm}
\section{Education}
\cventry{2019 - Now}{Ph.D. in Computer Science}{NYU  Courant}{NYC}{}{}
\cventry{2017 - 2019}{Masters in Operations Research}{Cornell}{Ithaca}{NY}{}
\cventry{2013 - 2017}{B.S. Physics}{Harvey Mudd College}{Claremont, CA}{GPA: 3.7}{}
%\cventry{2009 - 2013}{\textnormal{\emph{Georgetown Day School}}}{}{Washington, DC}{}{}

\vspace*{-1mm}

\section{Experience}

\cventry{2017 - Present}{PhD Student}{Andrew G. Wilson's lab}{NYU}{}{
Deep Learning: Building in inductive biases for structured data outside of images, video, text such as with
\begin{itemize}
\item Equivariance and Symmetries
\item Physical Priors and Dynamical Systems
\item Irregularly sampled spatial data
\item Probabilistic and Generative models
\end{itemize}
%Normalizing Flows
%\begin{itemize}
 %\item Developed Normalizing Flows for semi-supervised learning with competitive performance on text classification
 %\item Demonstrated how convolutional networks can be inverted directly and trained as normalizing flows
%\end{itemize}
%    Equivariant Convolutional Networks
%\begin{itemize}
%	\item Extending point convolutions to Lie groups, with applications to Chemistry and dynamical systems modeling
%	\item Preparing conference paper for submission.
%\end{itemize}
    }

%\cventry{2017 - 2019}{PhD Student}{Andrew Wilson's lab}{Cornell}{}{
%\begin{itemize}
%\item Obtained Masters degree and transferred to NYU
%\end{itemize}
%}
\vspace*{1.5mm} 
\cventry{Summer 2020}{Research Intern at Qualcomm}{with Max Welling}{}{Amsterdam, NL}{
	\begin{itemize}
		\item Developed probabilistic numeric convolutional neural networks, an approach that reasons about internal discretization errors probabilistically
		\item Project culminating in ICLR2021 Submission and patent application
	\end{itemize}
	}
\vspace*{1.5mm} 
\cventry{Summer 2019}{Applied Scientist intern at Amazon}{}{}{Seattle, WA}{
	\begin{itemize}
		\item Applying deep learning methods for ranking and recommendation
		\item Experience with models traditionally used for NLP such as LSTM and Transformer
	\end{itemize}
	}
%\vspace*{1.5mm} 
%\cventry{Spring 2017 -2019}{PhD Student}{Andrew G. Wilson's lab}{Cornell University}{}{
%Semi Supervised Learning
%\begin{itemize}
%        \item Achieved leading semi-supervised performance on CIFAR10 and CIFAR100, provided theoretical connection between consistency regularization and graph methods for SSL
%        \item Obtained Masters degree and transferred to NYU
%\end{itemize}
%}
\vspace*{1.5mm} 
\cventry{2015 - 2017}{Undergraduate Thesis in Physics}{Tom Donnelly's lab}{Harvey Mudd College}{}{
    \begin{itemize}
        %\item Led three-man HMC team to set up an experiment at UT Austin to test the theory of multipass Stochastic Heating using the high-power GHOST laser. This experiment required more than 20 optical elements inside the vacuum chamber, alignment to a $5\mu m$ focal spot size, and careful synchronization between two laser systems.
\item Led three-man HMC team at UT Austin to conduct laser physics experiment
		\item Applied computer vision to detect and register microspheres in SEM images, achieving 95\% accuracy.
        %\item LabView automation of laser experiments, using NiDAQs, and ThorLab components.
    \end{itemize}
    }

\vspace*{1.5mm} 
\cventry{Summers 2014, 2015}{Applied Physics Intern at NASA}{Alexander Kutyrev's lab}{NASA Goddard Space Flight Center}{}{
	\begin{itemize}
		\item Implemented a camera based image registration system to measure of mechanical positioning to sub-micron precision.
		\item Embedded systems programming, analogue and digital circuit design, PCB design
%\item Prototyped a control circuit and PCB to regulate the cryo temperature sensors control heating elements.
		%\item Brought concept of low cost, high precision (5mK), cryogenic temperature sensors based on the internal temperature response of commercially available transistors into practice.
%		\item Embedded systems programming in C\texttt{++} for PID control, interfacing with an external microcontroller over SPI.
	\end{itemize}
	}


 \section{Technical Skills}
 \cvline{Relevant Coursework}{Advanced Machine Learning Systems, Computer Vision,  Bayesian Machine Learning, Topics in ML optimization, Numerical Analysis for Data Science, Approximate Dynamic Programming, Algorithms, Stochastic Processes}
\cvline{Languages}{Python: 30k+ LoC, C\texttt{++}: 4k+ LoC, \LaTeX.}
\cvline{Reviewing}{AISTATS 2019, ICML 2019, NeurIPS 2019, ICLR 2020, NeurIPS 2020}
\vspace*{-2mm} 
\bibliographystyle{habbrvyr}
\bibliography{publications}
%\section*{Publications}
\renewcommand\refname{\begin{center}
                        \textbf{Reference}
                      \end{center}}

%\publication{flowgmm}%(\href{https://invertibleworkshop.github.io/accepted_papers/pdfs/INNF_2019_paper_28.pdf}{pdf})
\publication{ben2018}%(\href{https://openreview.net/pdf?id=rkgKBhA5Y7}{pdf})
       % \bibentry{ben2018}
       \publication{izmailov2019semi}
\publication{invertible}%(\href{https://invertibleworkshop.github.io/accepted_papers/pdfs/INNF_2019_paper_26.pdf}{pdf})
       \publication{finzi2020generalizing}
       \publication{finzi2020simplifying}
       \publication{benton2020augerino}
       \publication{finzi2021pncnn}
       \publication{finzi2021arbitrary}
\end{document}
%\section{Reviewing}

%\section{Teaching Experience} 

%\cventry{2017 - Present}{Teaching Assistant}{Cornell}{Ithaca, NY}{}{
%    TA for Simulation Modeling and Analysis\\
%    TA for Basic Engineering Probability and Statistics
%}
%\cventry{2015 - 2017}{Physics Academic Excellence Tutoring}{Harvey Mudd College}{Claremont, CA}{}{
%    Tutoring for freshman and sophomore physics classes.
%}

 %\vspace*{-20mm}   
 %\cvline{DL Frameworks}{Pytorch, Tensorflow}
% \cvline{Software}
% {GitHub, Eagle, Visual Studio.}
% \cvline{Hardware Experience}
% { \begin{itemize}
%         \item Analog PCB Design%
%         \item SPI, ADC's
 %       \end{itemize}
 %   }

%\section{Hobby Projects}

%\cvline{}{Chess Engine using MCTS and Deep Learning, trained on 4 Million Chess games labeled by Stockfish}
%\cvline{}{Traditional $\alpha-\beta$ search Chess Engine in C\texttt{++} w/ iterative deepening and transposition tables}
%\cvline{}{Interactive Numerical Schrodinger \& KdV equation simulators using Eigen and Split Step Fourier methods}
%\cvline{}{Graph-based circuit simulator supporting arbitrary RLC graphs with voltage and current sources}
%\cvline{}{Numerical N-Body gravitation simulator using symplectic integrators}

 %   \vspace*{-5.5mm}




%\vspace*{-1mm}
%\section{Extracurricular}
%\cventry{2014 - Present}{Programming Side Projects}{Claremont Colleges}{Claremont, CA}{}{
%	\begin{itemize}
%    \item A primitive Chess Engine written in C\texttt{++}
%	\item An interactive 1D Schrodinger Equation simulator in python to be used as a teaching tool
 %   \item A symplectic integrator for $n$ body gravitational system complete with 3D animation
%	\end{itemize}
%}



